\documentclass{scrreprt}[12pt,a4paper,twoside,duplex]
\usepackage[twoside,rmargin=1.8cm,lmargin=1.8cm,tmargin=2cm]{geometry}
\usepackage{scrjura}
\usepackage[utf8]{inputenc}
\usepackage[T1]{fontenc}
\usepackage[ngerman]{babel}
\usepackage{enumerate}
\usepackage{enumitem}
\usepackage{lmodern}
\usepackage{wasysym}
\usepackage[usenames, dvipsnames]{color}
\usepackage{multicol}
\setlength\columnsep{2em} %Spacing between columns

%Font
%\usepackage{avantgar}
\usepackage{bookman}
%\usepackage{ncntrsbk}

%\definecolor{zuBearbeiten}{RGB}{255, 0, 0}
% Wenn bearbeitet folgendes verwenden
\definecolor{zuBearbeiten}{RGB}{0, 0, 0}

\newcommand{\betriebskostenZuletztErmitteltAm}{\textcolor{zuBearbeiten}{XX.XX.20XX}}
\newcommand{\vertragsschlussDatum}{\textcolor{zuBearbeiten}{XX.XX.20XX}}
\newcommand{\vertragsschlussOrt}{\textcolor{zuBearbeiten}{Neuss}}

% BEGIN FORMAT
\tolerance 1414
\hbadness 1414
\emergencystretch 1.5em
\hfuzz 0.3pt
\widowpenalty=10000
\vfuzz \hfuzz
\raggedbottom
\sloppy
% END FORMAT

\begin{document}

\chapter*{Mietvertrag}

Zwischen §§vermieter.name§§, §§vermieter.addresse.strasse§§ §§vermieter.addresse.hausnummer§§, §§vermieter.addresse.plz§§ §§vermieter.addresse.stadt§§ als \textbf{Vermieter} und
\mbox{§§mieter.name§§}, derzeit wohnhaft in §§mieter.addresse.alt.strasse§§ §§mieter.addresse.alt.hausnummer§§, §§mieter.addresse.alt.plz§§ §§mieter.addresse.alt.stadt§§ als
\textbf{Mieter} wird folgender Mietvertrag vereinbart:

\begin{contract}
\Clause{title=Mietsache}
\label{mietsache:Raeume}

\begin{enumerate}
  \item Vermietet werden im Haus §§mieter.addresse.neu.strasse§§ §§mieter.addresse.neu.hausnummer§§ §§mieter.addresse.neu.stockwerk§§, folgende Räume:
\\\\
\textcolor{zuBearbeiten}{3 Zimmer, 1 Küche, 1 Flur/Diele, 1 Bad, 1 Balkon
\\\\
% \begin{itemize}
%   \item 2 Zimmer
%   \item 1 Kammer
%   \item 1 Küche
%   \item 1 Korridor/Diele
%   \item 1 Toilette mit Bad
%   \item 1 Balkon
%   \item 1 Kellerraum (Nr.\,5)
%   \item 1 PKW-Stellplatz (Nr.\,2.14)
% \end{itemize}
mit einer Wohnfläche von \textsl{67,43 $m^2$}}.
\item Die nachstehend aufgeführten Einrichtungen dürfen nach Maßgabe der
Benutzungsordnung mitbenutzt werden:

\begin{itemize}
  \color{zuBearbeiten}
  \item 1 kompl. Einbauküche
  \item 1 Einbauschrank Schlafzimmer
  \item 1 Einbauschrank Wohnzimmer
\end{itemize}
Das Mobiliar in Küche, Wohnzimmer und Schlafzimmer ist nicht Teil der Mietsache. Die Mieter leihen die Einbauschränke in Schlafzimmer, Wohnzimmer und Küche, sowie den Kühlschrank, den Herd und den Backofen für den häuslichen Gebrauch unentgeltlich.

Der Entleiher hat die gewöhnlichen Kosten der Erhaltung der geliehenen Sache zu tragen. Der Entleiher darf von der geliehenen Sache keinen anderen als den vertragsgemäßen Gebrauch machen. Er ist insbesondere nicht berechtigt, die Sache einem Dritten zu überlassen.
\item Der Vermieter verpflichtet sich, dem Mieter bei Übergabe der Mieträume
folgende Schlüssel auszuhändigen:
\begin{itemize}
  \color{zuBearbeiten}
  \item 2 Hausschlüssel
  \item 2 Hausbriefkastenschlüssel
\end{itemize}
Die Beschaffung weiterer Schlüssel durch den Mieter bedarf der Einwilligung des
Vermieters. Die Schlüssel sind bei Vertragsende zurückzugeben. Die nachgemachten Schlüssel sind den Vermietern kostenlos zu überlassen oder zu vernichten.
\item Die Mieträume dürfen vom Mieter nur zu Wohnzwecken genutzt werden. Die
Gesamtzahl der Personen, die die Wohnung beziehen werden beträgt \textsl{2}.
Der Mieter ist verpflichtet, seiner gesetzlichen Meldepflicht nachzukommen. Er versichert, dass er nicht beabsichtigt, eine weitere Person aufzunehmen. Die
Anbringung von Schildern, Werbung, Automaten und dergleichen außerhalb der
Mieträume bedarf der vorherigen schriftlichen Einwilligung des Vermieters.
\item Mit dem Mietvertrag verpflichten sich die Mieter eine Haftpflicht- und Hausratsversicherung abzuschließen, die die angemietete Einliegerwohnung einschließt.
\item Im Waschkeller dürfen die Mieter eine eigene Waschmaschine und Trockner aufbauen und anschließen. Das Trocknen der Wäsche im Waschkeller ist aufgrund des begrenzten Platzangebots nicht immer möglich.
\item Die Installation von Rauchwarnmeldern ist in den Bundesländern grundsätzlich gesetzlich vorgeschrieben. Der Mieter hat zu dulden, dass die Vermieter die Geräte einbaut, sie während der Mietzeit in einem ordnungsgemäßen Zustand erhält und dazu das Mietobjekt nach Absprache betritt.
\end{enumerate}
\end{contract}

\begin{contract}
\Clause{title=Zustand der Mieträume}
\begin{enumerate}
\item Der Mieter übernimmt die Mieträume im vorhandenen Zustand. Sofern ein Übergabeprotokoll erstellt wurde, wird auf dieses Bezug genommen.

Im Übrigen erkennt der Mieter die Räume als vertragsgerecht an. Dies gilt auch bezüglich des Renovierungszustandes der Räume. Das Bad weist bei Vermietung keine Gebrauchsspuren auf und ist mit geeigneten Reinigungsmitteln vom Mieter in gepflegtem Zustand zu halten. Einbohrlöcher in den Fliesen sind nicht gestattet. Das Sieb der Dusche muss vom Mieter regelmäßig gereinigt werden, da es sonst zu Wasserrückstau und Überschwemmungen kommen kann. Zudem muss insbesondere das Bad ausreichend gelüftet werden.
\item Das Parkett im Wohnzimmer ist mit Parkettreiniger zu reinigen und in gepflegtem Zustand zu halten.
\item Der Marmor im Treppenhaus darf nur mit Schmierseife feucht, aber niemals nass behandelt werden.
\item Die Vertragsparteien sind sich darüber einig, dass der Energieausweis nicht Bestandteil des Mietvertrages ist und die darin enthaltenen Angaben keine Zusicherung bestimmter Eigenschaften der Mietsache darstellen. Der Mieter kann aus den Angaben im Energieausweis keine Ansprüche oder gesonderten Rechte herleiten.
\end{enumerate}
\end{contract}

\begin{contract}
	\Clause{title=Beheizung und Warmwasserversorgung}
	\begin{enumerate}
		\item Die Vermieter halten die Heizungsanlage mindestens in der Zeit vom 1.10. bis zum 30.4. (Heizperiode) eines jeden Jahres in Betrieb, ansonsten soweit es die Witterung erfordert. Die Temperatur hat in der Zeit von 7.00 Uhr bis 23.00 Uhr mindestens 20 Grad Celsius zu betragen, es sei denn, der Mieter hat ein berechtigtes Interesse an einer anderweitigen Regelung.

		\item Der Umfang der Heizkosten bestimmt sich nach den gesetzlichen Vorschriften.

		\item Die Warmwasserversorgungsanlage haben die Vermieter ständig in Betrieb zu halten. Eine exzessive Nutzung kann verbrauchsbedingt zu einem Abbruch der Warmwasserzufuhr führen.
	\end{enumerate}
\end{contract}

\begin{contract}
\Clause{title=Mietzeit}

\begin{enumerate}
  \item\label{mietZeit:mietStart} Das Mietverhältnis beginnt am
  \textsl{§§mieter.miete.beginn§§}, es läuft auf \textbf{unbestimmte Zeit}. Kün\-di\-gungs\-fris\-ten siehe \textsl{\ref{mietZeit:fristen}}.
  \item\label{mietZeit:fristen} Kün\-di\-gungs\-fris\-ten zu
  \textsl{\ref{mietZeit:mietStart}}:\\\\
  Die Kün\-di\-gungs\-frist beträgt für den Mieter 3 Monate, für den Vermieter\\
  \begin{description}
    \item[3 Monate] wenn seit der Überlassung des Wohnraums weniger als 5 Jahre
    vergangen sind,
    \item[6 Monate] wenn seit der Überlassung des Wohnraums mehr als 5 Jahre vergangen
    sind,
    \item[9 Monate] wenn seit der Überlassung des Wohnraums mehr als 8 Jahre vergangen
    sind,
  \end{description}
  jeweils zum Ende eines Kalendermonats.
  \item Setzt der Mieter nach Ablauf des Mietverhältnisses den Gebrauch der Mietsache fort, wird das Mietverhältnis nicht stillschweigend verlängert oder neu begründet.
  \item Der Termin zur Abnahme der Wohnung ist zwei Wochen im Voraus zu vereinbaren. Die Räumlichkeiten sind in dem Zustand zurückzugeben, in dem sie übernommen werden. Dies impliziert, dass die Schränke ausgewischt übergeben werden. Andernfalls wird die Reinigung von Mobiliar und Inventar nach Aufwand bezahlt.
\end{enumerate}
\end{contract}

\begin{contract}
	\Clause{title=Personenmehrheit als Mieter}
	\begin{enumerate}
		\item Mehrere Mieter (z. B. Ehegatten) haften für alle Verpflichtungen aus dem Mietverhältnis als Gesamtschuldner.

		\item Mehrere Mieter bevollmächtigen sich gegenseitig zur Entgegennahme von Erklärungen der Vermieter sowie zur Abgabe eigener Erklärungen. Diese Bevollmächtigung gilt auch für die Entgegennahme von Kündigungen und Mieterhöhungsverlangen, nicht aber für die Abgabe von Kündigungserklärungen oder den Abschluss eines Mietaufhebungsvertrages.

		Mehrere Vermieter bevollmächtigen sich entsprechend.

		Bei Auszug eines Mieters von mehreren Mietern bleibt seine vertragliche Verpflichtung unberührt.
	\end{enumerate}
\end{contract}

\begin{contract}
\Clause{title=Miete und Nebenkosten}

\begin{enumerate}
  \item Die \textbf{Netto-Kaltmiete} (ausschließlich Betriebskosten) beträgt \textbf{§§mieter.miete.kalt§§\ §§mieter.miete.waehrung§§}.
  \item Neben der Miete sind monatlich der Betriebskostenvorschuss für Betriebskosten zu entrichten: \textbf{§§mieter.miete.nebenkosten§§\ §§mieter.miete.waehrung§§}
  \item \textbf{Insgesamt sind zzt.\ monatlich zu zahlen: §§mieter.miete.gesamt§§\ §§mieter.miete.waehrung§§}
  \item\label{betriebskosten} Die Betriebskosten gemäß Betriebskostenverordnung
  in der jeweils geltenden Fassung, ermittelt aufgrund der letzten Berechnung
  des Vermieters.\\
  \\
  Die Betriebskosten, insbesondere wie nachfolgend spezifiziert, sind als
  Vorschuss vom Mieter an den Vermieter zu zahlen. Die Abrechnung mit dem Mieter
  erfolgt jährlich. Die nachfolgende Spezifikation gilt auch bei Vereinbarung
  einer Betriebskostenpauschale.
  \begin{multicols}{2}
    \begin{enumerate}[label*=\arabic*\,)]
      \item Die laufenden öffentlichen Lasten des Grundstücks, insbesondere
      Grundsteuer
      \item Die Kosten der Stromversorgung
      \item Die Kosten der Wasserversorgung
      \item Die Kosten der Heizung und Warmwasserversorgung
      \item Die Kosten der Entwässerung (Ober\-flächen- und Sch\-mutz\-wasser)
      \item Die Kosten der Reinigung der Wege, Straßenreinigung und Müllbeseitigung
      \item Die Kosten der Ge\-bäu\-de\-rei\-ni\-gung und Ungezieferbekämpfung
      \item Die Kosten der Beleuchtung
      \item Die Kosten der Schornsteinreinigung
      \item Die Kosten der Sach- und Haft\-pflicht\-ver\-si\-che\-rung
      \item Die Kosten für den Hauswart
      \item Die Kosten der Dachrinnenreinigung
      \item Sonstige Betriebskosten
    \end{enumerate}
  \end{multicols}
  \item Umlagemaßstab: pro qm Wohnfläche

  Die gesamte Wohnfläche wird mit 143,25 m2 berechnet. Auf die Einliegerwohnung fallen darauf 67,43 m2 an. Der Multiplaktionswert wird deshalb mit 0.47 festgelegt.
  \item Die Betriebskosten werden nach dem Verhältnis der Wohn- bzw. Nutzflächen des Hauses umgelegt und berechnet.
  \item Über die Vorauszahlungen wird jährlich abgerechnet. Die Vermieter sind berechtigt, den Abrechnungszeitraum zu ändern, soweit dies zweckmäßig ist. Bei nachträglich erkannten Fehlern ist den Vermietern zur Korrektur berechtigt, auch wenn dies mit Mehrbelastungen für den Mieter verbunden ist.
  \item Die Vermieter behalten sich die Umlage aller umlegbaren Betriebskostenarten gem. der BetrKV vor, auch wenn sie längere Zeit nicht entstanden sind. Gleiches gilt, wenn sie zwar entstanden, aber nicht umgelegt wurden. Entstehen Betriebskosten nach Vertragsabschluss neu, können diese vom Vermieter auf den Mieter entsprechend der Umlage der übrigen Kosten verteilt werden. Der Vermieter ist befugt, eine angemessene Vorauszahlung festzusetzen.
  \item Im Falle einer Betriebskostenpauschale sind die Vermieter berechtigt, Erhöhungen der Betriebskosten entsprechend den gesetzlichen Bestimmungen geltend zu machen. Ermäßigen sich die Betriebskosten, ist er verpflichtet, die Pauschale herabzusetzen.
\end{enumerate}
\end{contract}

\begin{contract}
	\Clause{title=Sicherheitsleistung (Kaution)}

	\begin{enumerate}
		\item Der Mieter zahlt eine Mietsicherheit in Höhe von §§mieter.miete.kaution§§ §§mieter.miete.waehrung§§ (dreifache Miete ohne Betriebskosten).
		\item Die Kaution ist vor der Schlüsselübergabe auf das Konto der Vermieter zu überweisen.
		\item Die Vermieter können das Mietverhältnis aus wichtigem Grund kündigen, wenn der Mieter mit der Sicherheitsleistung in Höhe eines Betrages im Verzug ist, der der zweifachen Monatsmiete entspricht.
		\item Im Falle zulässiger Mieterhöhungen ist die Kaution entsprechend aufzustocken.
		\item Nimmt der Mieter mit Zustimmung der Vermieter bauliche Veränderungen vor,  ist er verpflichtet, eine zusätzliche Mietkaution zu zahlen, die angemessen ist, um die Wiederherstellung des ursprünglichen Zustandes zu sichern.
		\item Die Kaution ist nach Vertragsbeendigung und Rückgabe der Mietsache abzurechnen und an den Mieter auszuzahlen, sobald übersehbar ist, dass die Vermieter keine begründeten Gegenansprüche zustehen. Bei mehreren Mietern können die Vermieter die Kaution an jeden der Mieter zurückzahlen.
	\end{enumerate}
\end{contract}


\begin{contract}
\Clause{title=Zahlung der Kaution{,} Miete und der Nebenkosten}
  \begin{enumerate}
    \item Die Miete und Nebenkosten sind monatlich im Voraus, spätestens am 3.
    Werktag des Monats an den Vermieter zu zahlen.
    \item Die Kaution, Miete und die Nebenkosten sind auf das folgende Konto einzuzahlen:
    \begin{description}
      \item[Kontoinhaber] §§vermieter.name§§
      \item[Bank] §§vermieter.bankverbindung.name§§
      \item[IBAN] §§vermieter.bankverbindung.iban§§
      \item[BIC] §§vermieter.bankverbindung.bic§§
    \end{description}
    \item Bei verspäteter Zahlung kann der Vermieter Mahnkosten in Höhe von
    \textbf{§§mieter.miete.waehrung§§\ §§vermieter.miete.mahnkosten§§}\ je Mahnung, unbeschadet von
    Verzugszinsen, erheben. Bei Mahnkosten und Verzugszinsen handelt es sich um
    pauschalierten Schadensersatz.
    \item Empfängt der Mieter Sozialleistungen (z.B. Sozialhilfe oder Arbeitslosengeld II), ermächtigt er die Vermieter, die Leistungen unmittelbar vom Träger der Sozialleistungen einzufordern, soweit sie das Mietverhältnis betreffen.
  \end{enumerate}
\end{contract}

\begin{contract}
	\Clause{title=Benutzung der Mietsache}
	\begin{enumerate}
		\item Der Mieter darf die Mietsache ohne vorherige Einwilligung der Vermieter nicht zu einem anderen als dem vereinbarten Zweck nutzen.
		\item Der Mieter ist berechtigt, in den Mieträumen Haushaltsmaschinen aufzustellen, soweit dies die Kapazität der vorhandenen Installationen zulässt und die Statik des Hauses nicht beeinträchtigt wird. Führt der Anschluss einzelner Elektrogeräte zu einer Überlastung des vorhandenen Leitungsnetzes, ist der Mieter verpflichtet, die jeweiligen Geräte zu entfernen.
	\end{enumerate}
\end{contract}

\begin{contract}
	\Clause{title=Obhutspflicht{,} Instandhaltung{,} M\"angelanzeige}
	\begin{enumerate}
		\item Der Mieter verpflichtet sich zur sachgemäßen und pfleglichen Behandlung der Mietsache und der Räume und Flächen, die ihm zur Nutzung und Mitbenutzung zur Verfügung stehen. Er hat insbesondere für die ordnungsgemäße Reinigung sowie für die ausreichende Beheizung und Belüftung der Mieträume zu sorgen, um Kondenswasserschäden und ähnliche Schäden zu vermeiden.
		\item Die Anlagen und Einrichtungen in der Mietsache wie Schlösser, Ventile und Armaturen sind gängig zu halten. Die in den Mieträumen vorhandenen Wasser- und Abflussleitungen hat er vor dem Einfrieren zu schützen, soweit sie seiner unmittelbaren Einwirkung unterliegen. Bei längerer Abwesenheit ist für die Betreuung der Wohnung zu sorgen.
		\item Über auftretende Mängel der Mietsache hat der Mieter den Vermieter unverzüglich zu informieren. Ebenso, wenn der Mietsache oder dem Grundstück eine Gefahr droht.

		Der Mieter haftet für Schäden, die durch schuldhafte Verletzung der ihm obliegenden Obhuts- und Anzeigepflicht entstehen. Er haftet auch für das Verschulden von Familienangehörigen, Hausangestellten, Untermietern und sonstigen Personen, die sich mit seinem Willen in der Wohnung aufhalten.
	\end{enumerate}
\end{contract}

\begin{contract}
	\Clause{title=Schönheitsreparaturen}
	\begin{enumerate}
		\item Die Mieter übernehmen auf eigene Kosten laufende – turnusmäßig wiederkehrende – Schönheitsreparaturen.
		\item Die Schönheitsreparaturen umfassen das Tapezieren und Anstreichen der Wände. Naturlasiertes Holzwerk und Kunststoffrahmen dürfen nicht mit Deckfarbe überstrichen werden. Alle Schönheitsreparaturen sind fachgerecht auszuführen.

		Das Rauchen in den Mieträumen ist grundsätzlich untersagt. Der Mieter kann Schadenersatz wegen Nichterfüllung verlangen.
	\end{enumerate}
\end{contract}

\begin{contract}
	\Clause{title=Kleine Instandhaltungen}
	\begin{enumerate}
		\item Der Mieter hat verschuldensunabhängig die Kosten zu tragen für kleinere Instandhaltungs- und Instandsetzungsmaßnahmen an den Installationsgegenständen für Elektrizität, Wasser und Gas, den Heiz- und Kocheinrichtungen, den Fenster- und Türverschlüssen, Rollläden sowie den Verschlussvorrichtungen von Fensterläden.

		Elektrogeräte der Eibauküche gehören nicht zur Mietsache und müssen vom Mieter selbst repariert werden.
	\end{enumerate}
\end{contract}

\begin{contract}
	\Clause{title=Reinigungspflicht}
	\begin{enumerate}
		\item Der Mieter übernimmt auch die Reinigung der gemeinsam benutzten Räume, Treppen, Flure, Fenster sowie der Zuwege zum Haus. Zur Reinigung der Zuwege gehört auch der Winterdienst im Sinne der nachstehenden Ziffer 2.

		Der Mieter ist verpflichtet, sich an der regelmäßigen Reinigung von gemeinschaftlich benutzten Räumen, Einrichtungen, Wegen und Zufahrten (beispielsweise Eingänge, Flure, Treppenhaus, Kellerräume, Abstellplatz für Mülltonnen) in angemessenem und in dem Haus üblichen Umfang zu beteiligen.

		Bei nicht ordnungsgemäßer Reinigung lassen die Vermieter die erforderlichen Arbeiten auf Kosten des Mieters anderweitig ausführen. Jedoch erst dann, wenn die Abmahnung erfolglos geblieben ist. Soweit der Mieter die Reinigungsarbeiten leistet, entfällt seine anteilige Belastung an den Kosten der Hausreinigung.

		\item Der Mieter übernimmt abwechselnd die Reinigung des Zuweges und der Wege rund um das Haus. Bei Glätte ist mit abstumpfenden Mitteln – falls notwendig, wiederholt – zu streuen. Tausalz- und tausalzhaltige Mittel dürfen nicht verwendet werden. Schnee ist unverzüglich nach Beendigung des Schneefalls zu räumen. Bei Glatteis ist sofort zu streuen.

		Ist der Mieter persönlich verhindert (z. B. Urlaub, Krankheit usw.), hat er auf eigene Kosten dafür zu sorgen, dass die Arbeiten anderweitig durchgeführt werden.
	\end{enumerate}
\end{contract}

\begin{contract}
	\Clause{title=Tierhaltung}
	\begin{enumerate}
		\item Der Mieter darf ohne schriftliche Zustimmung des Vermieters kleinere Tiere (z.B. Ziervögel und Zierfische) in den Wohnräumen halten, sofern sich die Anzahl der Tiere in den üblichen, artgerechten Grenzen hält und soweit nach der Art der Tiere und ihrer Unterbringung Belästigungen von Hausbewohnern und Nachbarn sowie Beeinträchtigungen der Mietsache und/oder des Grundstücks nicht zu erwarten sind.
		\item Jede darüber hinausgehende Tierhaltung (z.B. Hund, Katze) innerhalb der Mietwohnung bedarf der vorherigen Zustimmung des Vermieters. Hierbei hat er seine Interessen mit denjenigen des Mieters und evtl. weiterer beteiligter Personen umfassend abzuwägen. Die Zustimmung erfolgt nach pflichtgemäßen Ermessen und wird nur für den Einzelfall erteilt und kann widerrufen werden, wenn ein wichtiger Grund vorliegt. Der Widerruf kommt in Betracht, wenn der Mieter Auflagen nicht einhält, Hausbewohner oder Nachbarn belästigt oder wenn die Mietsache oder das Grundstück beeinträchtigt werden.
		\item Die einschlägigen Tierschutzgesetze sind zu beachten. Hunde, die nach der Hundeverordnung als „Kampfhunde“ gelten, dürfen nicht gehalten werden. Eine erteilte Zustimmung gilt nur bis zum Tode oder zur Abschaffung des Tieres. Schafft der Mieter ein Tier neu an, bedarf es hierzu wiederum der Zustimmung des Vermieters.
		\item Dem Mieter ist es nicht gestattet, von seiner Wohnung aus oder auf dem Grundstück Tauben oder andere Tiere zu füttern.
	\end{enumerate}
\end{contract}

\begin{contract}
	\Clause{title=Empfangsanlagen für Rundfunk und Fernsehen}
	\begin{enumerate}
		\item Der Mieter ist jederzeit berechtigt, die Hochantenne zu nutzen. Verfügt das Haus über einen Breitbandkabelanschluss, ist der Mieter befugt, hiervon Gebrauch zu machen. Hingegen darf er eine CB-Dachfunkantenne nur mit Zustimmung des Vermieters installieren.
		\item Die Montage einer Parabolantenne ist dem Mieter nicht gestattet.
		\item Der Mieter stellt den Vermieter von allen im Zusammenhang mit der Installation der Antenne entstehenden Kosten und Gebühren frei.
	\end{enumerate}
\end{contract}

\begin{contract}
	\Clause{title=Bauliche Veränderungen{,}  Modernisierung}
	\begin{enumerate}
		\item Maßnahmen zur Verbesserung der gemieteten Räume oder sonstiger Teile des Gebäudes, zur Einsparung von Energie oder Wasser oder zur Schaffung neuen Wohnraums hat der Mieter im gesetzlichen Umfang zu dulden. Er hat die in Betracht kommenden Räume nach vorheriger Terminabsprache zugänglich zu halten. Behindert oder verzögert er schuldhaft die Arbeiten, hat er den Schaden zu ersetzen.
		\item     2. Der Mieter darf Instandsetzungen jeglicher Art, bauliche oder sonstige Änderungen (z. B. Einbau von Einrichtungen) nur durchführen, wenn die Vermieter vorher schriftlich einwilligen. Bei eigenmächtigem Handeln des Mieters sind die Vermieter zur Übernahme der Kosten nicht verpflichtet.

		Bei baulichen oder sonstigen Änderungen (z. B. Einbau von Einrichtungen), die der Mieter ohne Zustimmung des Vermieters vornimmt, hat er auf eigene Kosten den früheren Zustand wiederherzustellen, wenn die Vermieter dies verlangen. Im Weigerungsfall sind die Vermieter berechtigt, die Beseitigung auf Kosten des Mieters vornehmen zu lassen.

		Bei baulichen Änderungen des Mieters, die er mit Zustimmung des Vermieters vornimmt, können die Vermieter verlangen, dass der Mieter beim Auszug auf eigene Kosten den früheren Zustand wiederherstellt.
		\item Tür- und Briefkastenschilder bringen die Vermieter auf Kosten des Mieters an; der Mieter trägt auch die Kosten für die Entfernung.
	\end{enumerate}
\end{contract}

\begin{contract}
	\Clause{title=Haftung des Vermieters{,} Aufrechnung und Zurückbehaltung}
	\begin{enumerate}
		\item Führt ein Mangel an der Mietsache zu Sach- und Vermögensschäden des Mieters, haften die Vermieter nur, soweit er dies zu vertreten hat (z. B. Vorsatz und grobe Fahrlässigkeit).
		\item Zur Aufrechnung mit Gegenforderungen gegenüber der Mietforderung oder zur Ausübung eines Zurückbehaltungsrechts ist der Mieter außer im Fall unbestrittener oder rechtskräftig festgestellter Forderungen nur berechtigt, wenn er diese Absicht mindestens einen Monat vor Fälligkeit dem Vermieter schriftlich angezeigt hat.
	\end{enumerate}
\end{contract}

\begin{contract}
	\Clause{title=Betreten der Mieträume durch den Vermieter}
	\begin{enumerate}
		\item Die Vermieter oder von ihnen Beauftragte dürfen die Mieträume aus begründetem konkreten Anlass (z.B. zur Prüfung  ihres Zustandes, zum Ablesen der Messgeräte oder bei Besichtigung durch Nachmieter, Reparaturen) in angemessenem Umfang und nach rechtzeitiger Vorankündigung betreten. Auf eine persönliche Verhinderung des Mieters ist Rücksicht zu nehmen.
		\item Beabsichtigen die Vermieter das Grundstück und/oder die Mietwohnung zu verkaufen oder ist der Mietvertrag gekündigt, sind die  Vermieter oder von ihm Beauftragte berechtigt, die Mietsache nach rechtzeitiger Vorankündigung zusammen mit Kauf- oder Mietinteressenten zu besichtigen. Der Vermieter hat sein Recht so schonend wie möglich auszuüben.
		\item Bei längerer Abwesenheit hat der Mieter sicherzustellen, dass die Rechte des Vermieters im vorstehenden Sinn ausgeübt werden können. Der Mieter hat die Schlüssel gegebenenfalls bei einer Vertrauensperson zu hinterlegen.
		\item Bei Gefahr oder Schäden im Verzug ist der Vermieter oder ein vom ihm Beauftragter berechtigt, die Mieträume zur Durchführung der zur Abwehr der Gefahr notwendigen Arbeiten zu betreten.
	\end{enumerate}
\end{contract}

\begin{contract}
	\Clause{title=Außerordentliches Kündigungsrecht des Vermieters} 
	
	\begin{enumerate}
	\item Die Vermieter können das Mietverhältnis ohne Einhaltung einer Kündigungsfrist kündigen, wenn der Mieter für zwei aufeinander folgende Termine oder mit mehr als einer Monatsmiete in Zahlungsverzug ist. Das Gleiche gilt, wenn er in einem Zeitraum, der sich über mehr als zwei Monate erstreckt, mit einem Betrag in Verzug ist, der die Miete für zwei Monate erreicht.
	
	\item Der Mietrückstand umfasst auch die monatlichen Betriebskostenvorauszahlungen und Pauschalen sowie Untermietzuschläge, nicht jedoch Nachforderungen aus der Betriebskostenabrechnung.
	\end{enumerate}
\end{contract}

\begin{contract}
	\Clause{title=Schriftform}
	Mündliche Nebenabreden bestehen nicht. Soweit nachträgliche Änderungen und Ergänzungen in Rede stehen, wird der Mietvertrag entsprechend ergänzt. Es bedarf der Schriftform.
\end{contract}

\begin{contract}
	\Clause{title=Zusätzliche Vereinbarungen}
	Aufgrund des engen Beisammenlebens wünschen die Vermieter sich eine transparente Kommunikation.
\end{contract}

\begin{contract}
	\Clause{title=Datenschutz}
	\begin{enumerate}
		\item Die in diesem Mietvertrag erhobenen personenbezogenen Daten werden vom Vermieter benötigt, um sicherzustellen, dass dieser gem. Art.6 Abs.1 S.1 lit. b und f DSGVO seine Verpflichtungen aus dem Mietverhältnis und dessen Abwicklung erfüllen kann. Wegen der weiteren Einzelheiten wird auf die Datenschutzinformation in Anhang 4 verwiesen.
		\item Der Mietvertrag bzw. dessen Vertragstext und die darin enthaltenen Daten werden vom Vermieter gegebenenfalls auch elektronisch verarbeitet und gespeichert. Der Mietvertrag kann auch auf einer vom Herausgeber des Mietvertrags-Formulars betriebenen oder lizenzierten Internetplattform gespeichert und archiviert werden. Mit Ausnahme von IT-Dienstleistern, deren Dienste für den reibungslosen Betrieb zwingend erforderlich sind, ist ein Zugriff durch Dritte ausgeschlossen. Eine Auswertung, Weitergabe, Aggregierung oder sonstige Verarbeitung der Daten findet nicht statt.
		\item Der Mieter ist damit einverstanden, dass Daten über die Miethöhe sowie über Art, Größe, Ausstattung, Beschaffenheit und Lage der Wohnung gespeichert und an Dritte weitergegeben werden dürfen, um Mietpreisübersichten und Vergleichsmietensammlungen zu erstellen. Der Vermieter versichert, dass die Daten vertraulich behandelt und ausschließlich zu diesen Zwecken verwendet werden. Der Vermieter ist auf Verlangen des Mieters verpflichtet, ihm Auskunft über die Personen und Stellen zu erteilen, an die die Daten übermittelt werden.
		\item Der Mieter erteilt ausdrücklich seine Einwilligung, dass der Vermieter die Energieverbrauchsdaten der Mietsache z.B. zum Zwecke der Erstellung eines gesetzlich vorgeschriebenen Energieausweises direkt beim Energieversorger abfragen darf. Dies gilt insbesondere für den Fall, dass der Mieter die Energieversorgung direkt von dem Anbieter bezieht.
		\item Verweigert oder widerruft der Mieter seine Einwilligung, lässt dies den Bestand des Mietvertrages unberührt.
	\end{enumerate}
\end{contract}

\begin{contract}
	\Clause{title=Beendigung des Mietverhältnisses}
	\begin{enumerate}
		\item Der Mieter hat die Mietsache beim Auszug vollständig geräumt und gereinigt an die Vermieter zurückzugeben. Er hat ihm sämtliche Schlüssel auszuhändigen; diejenigen Schlüssel, die der Mieter zusätzlich auf seine Kosten hat anfertigen lassen, hat er dem Vermieter kostenlos zu überlassen oder ihre Vernichtung nachzuweisen.
		\item Der Mieter hat beim Auszug seine neue Anschrift bekannt zu geben. Außerdem ist er verpflichtet, dem Vermieter seine Abmeldebescheinigung vorzulegen; im Falle der unzulässigen Untervermietung hat er dem Vermieter auch die Abmeldebescheinigung des Untermieters vorzulegen.
		\item Einrichtungen, mit denen der Mieter die Räume versehen hat, darf er wegnehmen. Der Vermieter kann die Ausübung des Wegnahmerechts durch Zahlung einer angemessenen Entschädigung abwenden, es sei denn, der Mieter hat ein berechtigtes Interesse an der Wegnahme.

		Auf Verlangen des Vermieters hat der Mieter bauliche Veränderungen oder eingebaute Einrichtungen bei Ende des Mietvertrages auf seine Kosten wegzunehmen und den ursprünglichen Zustand wiederherzustellen, sofern nichts anderes schriftlich vereinbart wurde.
		\item Bei verspäteter Rückgabe der Mietsache hat der Mieter als Entschädigung für die Dauer der Vorenthaltung die für vergleichbare Mietsachen ortsübliche Miete zu entrichten. Maßgeblich ist dabei die bei Neuabschluss eines Mietvertrages über die Wohnung ortsübliche Miete (Marktmiete).

		Räumt der Mieter zur Unzeit, hat er die Entschädigung für den vollen Monat zu leisten. Die Geltendmachung eines weiteren Schadens ist nicht ausgeschlossen, wenn die Rückgabe infolge von Umständen unterbleibt, die der Mieter zu vertreten hat.
		\item Zieht der Mieter aus, ohne dem Vermieter seine neue Anschrift oder seinen neuen Aufenthaltsort bekannt zu geben, ist der Vermieter berechtigt, die Wohnung in Besitz zu nehmen und die Schlösser oder Schließzylinder auszuwechseln. Der Vermieter ist befugt, zurückgelassene Gegenstände zu vernichten, die augenscheinlich wertlos oder nur von geringem Wert sind.
	\end{enumerate}
\end{contract}

\begin{contract}
	\Clause{title=Wirksamkeit der Vertragsbestimmungen}
	Durch etwaige Ungültigkeit einer Bestimmung oder mehrerer Bestimmungen dieses Vertrages wird die Gültigkeit der übrigen Bestimmungen nicht berührt.
	Dieser Vertrag ist doppelt und gleich lautend ausgefertigt, selbst gelesen, genehmigt und eigenhändig unterschrieben. Jede Partei erhält eine Ausfertigung (Ehegatten nur ein Exemplar).
\end{contract}

\newpage

§§mieter.vertragsschluss.ort§§, den §§mieter.vertragsschluss.datum§§

    \

    \

    \

    \

    \



    \hrule\vspace{1ex} Vermieter

    \

    \

    \

    \

    \hrule\vspace{1ex} Mieter


    \end{document}